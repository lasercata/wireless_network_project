\documentclass[a4paper, 12pt, twoside]{article}


%------------------------------------------------------------------------
%
% Author                :   Lasercata
% Last modification     :   2025.05.21
%
%------------------------------------------------------------------------

%---------Init {{{1
%------Lang
% \usepackage[french]{babel}
%\usepackage[english]{babel}

\input{data/data.sty}


%See https://github.com/lasercata/LaTeX_Templates for the file latex_base.sty
% \documentclass[a4paper, 12pt, twoside]{article}


%------------------------------------------------------------------------
%
% Author                :   Lasercata
% Last modification     :   2025.05.07
%
%------------------------------------------------------------------------

%---------Init {{{1
%------Lang
\usepackage[french]{babel}
%\usepackage[english]{babel}


%See https://github.com/lasercata/LaTeX_Templates for the file latex_base.sty
% \documentclass[a4paper, 12pt, twoside]{article}


%------------------------------------------------------------------------
%
% Author                :   Lasercata
% Last modification     :   2025.05.07
%
%------------------------------------------------------------------------

%---------Init {{{1
%------Lang
\usepackage[french]{babel}
%\usepackage[english]{babel}


%See https://github.com/lasercata/LaTeX_Templates for the file latex_base.sty
% \documentclass[a4paper, 12pt, twoside]{article}


%------------------------------------------------------------------------
%
% Author                :   Lasercata
% Last modification     :   2025.05.07
%
%------------------------------------------------------------------------

%---------Init {{{1
%------Lang
\usepackage[french]{babel}
%\usepackage[english]{babel}


%See https://github.com/lasercata/LaTeX_Templates for the file latex_base.sty
% \input{~/Templates/latex_base.sty}
\input{latex_base.sty}


%------Circuitikz
%\usetikzlibrary{babel}             %Uncomment this to use circuitikz
%\usetikzlibrary{shapes.geometric}  % To draw triangles in trees
%\usepackage[european]{circuitikz}            %Electrical circuits drawing

%------Sections
%---To change section numbering :
% \renewcommand\thesection{\Roman{section}}
% \renewcommand\thesubsection{\arabic{subsection})}
% \renewcommand\thesubsubsection{\textit \alph{subsubsection})}

%---To start numbering sections from 0
% \setcounter{section}{-1}

%---To hide subsubsection from the table of contents (show with max depth of 2)
% \setcounter{tocdepth}{2}


%------Logo
% \setlogo[pics/logo.png] %Comment to remove the logo
\fancyfoot[LO, RE]{\includegraphics[scale=0.05]{pics/logo.png}}
%}}}1

%------Title (with default LaTeX style)
\title{Projet JEE -- Rapport}
\author{
    Équipe Tonnerre de Zeus
    \\ Laszlo \textsc{Abadie}
    \\ Adem \textsc{Benkeddad}
    \\ Samy \textsc{Chaabi}
    \\ Sami \textsc{Taki}
    \\ Louis \textsc{Thomas-Girardey}
}
\date{
    \today
    \\
    $\phantom{a}$
    \\
    \includegraphics[width=.6\textwidth]{pics/logo.png}
}

%---------------------------------Begin Document
\begin{document}
    
    % Title {{{1
    % \thetitle{Projet d'algorithmique avancée}{Rapport}
    \maketitle

    \newpage
    
    \tableofcontents
    \listoffigures
    \listoftables
    \listofalgorithms
    \newpage
    % }}}1

    \begin{indt}{\section{Structure du projet}} %{{{1
        \begin{indt}{\subsection{Arborescence}} %{{{2
            \begin{figure}[H]
                \centering
            
                \includegraphics[scale=.7]{pics/tree_0.png}
            
                \caption{Arborescence du projet (fichiers java)}
            \end{figure}

            Le projet est découpé en plusieurs packages java.
        \end{indt} %}}}2

        \begin{indt}{\subsection{API}} %{{{2
            \begin{indt}{Nous avons choisi d'implémenter une API REST suivante, avec trois points d'entrée :}
                $-$ \texttt{/api/attractions} ;

                $-$ \texttt{/api/restaurants} ;

                $-$ \texttt{/api/shows} ;
            \end{indt}

            Avec par exemple pour les attractions :

            \begin{lstlisting}[xleftmargin=60pt]
GET /api/attractions         - Retreives a list of all attractions
GET /api/attractions/{id}    - Retreives data for the attraction of id `id`
POST /api/attractions        - Creates a new attraction
PUT /api/attractions/{id}    - Updates an existing attraction
DELETE /api/attractions/{id} - Deletes an existing attraction \end{lstlisting}
        \end{indt} %}}}2
    \end{indt} %}}}1

    \begin{indt}{\section{Choix d'implémentation}} %{{{1
        Il y a trois classes principales : \texttt{Attraction}, \texttt{Restaurant} et \texttt{Show}.
        Ces classes représentent respectivement les attractions, restaurants et les spectacles.

        En plus de ces classes, on a pour chacune une classe DAOImpl, qui contient des méthodes permettant d'échanger avec la base de données.
        Il y a également pour chaque classe une classe Controller qui permet de créer les endpoints.

        Il y a une interface \texttt{GeneralDAO} qui est paramétrée par un type (qui sera une des classes), qui est implémentée par les DAOImpl.
        Elle contient aussi des méthodes statiques.

        \vspace{12pt}
        
        Nous avons choisi de d'implémenter la logique (\textit{i.e} les vérifications d'horaires, de chevauchement, etc.) dans des triggers sur la base de donnée.

        \vspace{12pt}
        
        Pour le frontend, nous avons factorisé le code javascript de génération des pages HTML, comme elles sont très similaires.
    \end{indt} %}}}1

    \begin{indt}{\section{État du projet}} %{{{1
        Nous avons un backend fonctionnel : exposition de l'API, communication entre le java et le SQL, les triggers.

        Le frontend est partiellement fonctionnel, mais ne communique pas bien avec le backend.
    \end{indt} %}}}1
    
\end{document}
%--------------------------------------------End

% vim:foldmethod=marker:foldlevel=0
\documentclass[a4paper, 12pt, twoside]{article}


%------------------------------------------------------------------------
%
% Author                :   Lasercata
% Last modification     :   2025.05.07
%
%------------------------------------------------------------------------

%---------Init {{{1
%------Lang
\usepackage[french]{babel}
%\usepackage[english]{babel}


%See https://github.com/lasercata/LaTeX_Templates for the file latex_base.sty
% \input{~/Templates/latex_base.sty}
\input{latex_base.sty}


%------Circuitikz
%\usetikzlibrary{babel}             %Uncomment this to use circuitikz
%\usetikzlibrary{shapes.geometric}  % To draw triangles in trees
%\usepackage[european]{circuitikz}            %Electrical circuits drawing

%------Sections
%---To change section numbering :
% \renewcommand\thesection{\Roman{section}}
% \renewcommand\thesubsection{\arabic{subsection})}
% \renewcommand\thesubsubsection{\textit \alph{subsubsection})}

%---To start numbering sections from 0
% \setcounter{section}{-1}

%---To hide subsubsection from the table of contents (show with max depth of 2)
% \setcounter{tocdepth}{2}


%------Logo
% \setlogo[pics/logo.png] %Comment to remove the logo
\fancyfoot[LO, RE]{\includegraphics[scale=0.05]{pics/logo.png}}
%}}}1

%------Title (with default LaTeX style)
\title{Projet JEE -- Rapport}
\author{
    Équipe Tonnerre de Zeus
    \\ Laszlo \textsc{Abadie}
    \\ Adem \textsc{Benkeddad}
    \\ Samy \textsc{Chaabi}
    \\ Sami \textsc{Taki}
    \\ Louis \textsc{Thomas-Girardey}
}
\date{
    \today
    \\
    $\phantom{a}$
    \\
    \includegraphics[width=.6\textwidth]{pics/logo.png}
}

%---------------------------------Begin Document
\begin{document}
    
    % Title {{{1
    % \thetitle{Projet d'algorithmique avancée}{Rapport}
    \maketitle

    \newpage
    
    \tableofcontents
    \listoffigures
    \listoftables
    \listofalgorithms
    \newpage
    % }}}1

    \begin{indt}{\section{Structure du projet}} %{{{1
        \begin{indt}{\subsection{Arborescence}} %{{{2
            \begin{figure}[H]
                \centering
            
                \includegraphics[scale=.7]{pics/tree_0.png}
            
                \caption{Arborescence du projet (fichiers java)}
            \end{figure}

            Le projet est découpé en plusieurs packages java.
        \end{indt} %}}}2

        \begin{indt}{\subsection{API}} %{{{2
            \begin{indt}{Nous avons choisi d'implémenter une API REST suivante, avec trois points d'entrée :}
                $-$ \texttt{/api/attractions} ;

                $-$ \texttt{/api/restaurants} ;

                $-$ \texttt{/api/shows} ;
            \end{indt}

            Avec par exemple pour les attractions :

            \begin{lstlisting}[xleftmargin=60pt]
GET /api/attractions         - Retreives a list of all attractions
GET /api/attractions/{id}    - Retreives data for the attraction of id `id`
POST /api/attractions        - Creates a new attraction
PUT /api/attractions/{id}    - Updates an existing attraction
DELETE /api/attractions/{id} - Deletes an existing attraction \end{lstlisting}
        \end{indt} %}}}2
    \end{indt} %}}}1

    \begin{indt}{\section{Choix d'implémentation}} %{{{1
        Il y a trois classes principales : \texttt{Attraction}, \texttt{Restaurant} et \texttt{Show}.
        Ces classes représentent respectivement les attractions, restaurants et les spectacles.

        En plus de ces classes, on a pour chacune une classe DAOImpl, qui contient des méthodes permettant d'échanger avec la base de données.
        Il y a également pour chaque classe une classe Controller qui permet de créer les endpoints.

        Il y a une interface \texttt{GeneralDAO} qui est paramétrée par un type (qui sera une des classes), qui est implémentée par les DAOImpl.
        Elle contient aussi des méthodes statiques.

        \vspace{12pt}
        
        Nous avons choisi de d'implémenter la logique (\textit{i.e} les vérifications d'horaires, de chevauchement, etc.) dans des triggers sur la base de donnée.

        \vspace{12pt}
        
        Pour le frontend, nous avons factorisé le code javascript de génération des pages HTML, comme elles sont très similaires.
    \end{indt} %}}}1

    \begin{indt}{\section{État du projet}} %{{{1
        Nous avons un backend fonctionnel : exposition de l'API, communication entre le java et le SQL, les triggers.

        Le frontend est partiellement fonctionnel, mais ne communique pas bien avec le backend.
    \end{indt} %}}}1
    
\end{document}
%--------------------------------------------End

% vim:foldmethod=marker:foldlevel=0


%------Circuitikz
%\usetikzlibrary{babel}             %Uncomment this to use circuitikz
%\usetikzlibrary{shapes.geometric}  % To draw triangles in trees
%\usepackage[european]{circuitikz}            %Electrical circuits drawing

%------Sections
%---To change section numbering :
% \renewcommand\thesection{\Roman{section}}
% \renewcommand\thesubsection{\arabic{subsection})}
% \renewcommand\thesubsubsection{\textit \alph{subsubsection})}

%---To start numbering sections from 0
% \setcounter{section}{-1}

%---To hide subsubsection from the table of contents (show with max depth of 2)
% \setcounter{tocdepth}{2}


%------Logo
% \setlogo[pics/logo.png] %Comment to remove the logo
\fancyfoot[LO, RE]{\includegraphics[scale=0.05]{pics/logo.png}}
%}}}1

%------Title (with default LaTeX style)
\title{Projet JEE -- Rapport}
\author{
    Équipe Tonnerre de Zeus
    \\ Laszlo \textsc{Abadie}
    \\ Adem \textsc{Benkeddad}
    \\ Samy \textsc{Chaabi}
    \\ Sami \textsc{Taki}
    \\ Louis \textsc{Thomas-Girardey}
}
\date{
    \today
    \\
    $\phantom{a}$
    \\
    \includegraphics[width=.6\textwidth]{pics/logo.png}
}

%---------------------------------Begin Document
\begin{document}
    
    % Title {{{1
    % \thetitle{Projet d'algorithmique avancée}{Rapport}
    \maketitle

    \newpage
    
    \tableofcontents
    \listoffigures
    \listoftables
    \listofalgorithms
    \newpage
    % }}}1

    \begin{indt}{\section{Structure du projet}} %{{{1
        \begin{indt}{\subsection{Arborescence}} %{{{2
            \begin{figure}[H]
                \centering
            
                \includegraphics[scale=.7]{pics/tree_0.png}
            
                \caption{Arborescence du projet (fichiers java)}
            \end{figure}

            Le projet est découpé en plusieurs packages java.
        \end{indt} %}}}2

        \begin{indt}{\subsection{API}} %{{{2
            \begin{indt}{Nous avons choisi d'implémenter une API REST suivante, avec trois points d'entrée :}
                $-$ \texttt{/api/attractions} ;

                $-$ \texttt{/api/restaurants} ;

                $-$ \texttt{/api/shows} ;
            \end{indt}

            Avec par exemple pour les attractions :

            \begin{lstlisting}[xleftmargin=60pt]
GET /api/attractions         - Retreives a list of all attractions
GET /api/attractions/{id}    - Retreives data for the attraction of id `id`
POST /api/attractions        - Creates a new attraction
PUT /api/attractions/{id}    - Updates an existing attraction
DELETE /api/attractions/{id} - Deletes an existing attraction \end{lstlisting}
        \end{indt} %}}}2
    \end{indt} %}}}1

    \begin{indt}{\section{Choix d'implémentation}} %{{{1
        Il y a trois classes principales : \texttt{Attraction}, \texttt{Restaurant} et \texttt{Show}.
        Ces classes représentent respectivement les attractions, restaurants et les spectacles.

        En plus de ces classes, on a pour chacune une classe DAOImpl, qui contient des méthodes permettant d'échanger avec la base de données.
        Il y a également pour chaque classe une classe Controller qui permet de créer les endpoints.

        Il y a une interface \texttt{GeneralDAO} qui est paramétrée par un type (qui sera une des classes), qui est implémentée par les DAOImpl.
        Elle contient aussi des méthodes statiques.

        \vspace{12pt}
        
        Nous avons choisi de d'implémenter la logique (\textit{i.e} les vérifications d'horaires, de chevauchement, etc.) dans des triggers sur la base de donnée.

        \vspace{12pt}
        
        Pour le frontend, nous avons factorisé le code javascript de génération des pages HTML, comme elles sont très similaires.
    \end{indt} %}}}1

    \begin{indt}{\section{État du projet}} %{{{1
        Nous avons un backend fonctionnel : exposition de l'API, communication entre le java et le SQL, les triggers.

        Le frontend est partiellement fonctionnel, mais ne communique pas bien avec le backend.
    \end{indt} %}}}1
    
\end{document}
%--------------------------------------------End

% vim:foldmethod=marker:foldlevel=0
\documentclass[a4paper, 12pt, twoside]{article}


%------------------------------------------------------------------------
%
% Author                :   Lasercata
% Last modification     :   2025.05.07
%
%------------------------------------------------------------------------

%---------Init {{{1
%------Lang
\usepackage[french]{babel}
%\usepackage[english]{babel}


%See https://github.com/lasercata/LaTeX_Templates for the file latex_base.sty
% \documentclass[a4paper, 12pt, twoside]{article}


%------------------------------------------------------------------------
%
% Author                :   Lasercata
% Last modification     :   2025.05.07
%
%------------------------------------------------------------------------

%---------Init {{{1
%------Lang
\usepackage[french]{babel}
%\usepackage[english]{babel}


%See https://github.com/lasercata/LaTeX_Templates for the file latex_base.sty
% \input{~/Templates/latex_base.sty}
\input{latex_base.sty}


%------Circuitikz
%\usetikzlibrary{babel}             %Uncomment this to use circuitikz
%\usetikzlibrary{shapes.geometric}  % To draw triangles in trees
%\usepackage[european]{circuitikz}            %Electrical circuits drawing

%------Sections
%---To change section numbering :
% \renewcommand\thesection{\Roman{section}}
% \renewcommand\thesubsection{\arabic{subsection})}
% \renewcommand\thesubsubsection{\textit \alph{subsubsection})}

%---To start numbering sections from 0
% \setcounter{section}{-1}

%---To hide subsubsection from the table of contents (show with max depth of 2)
% \setcounter{tocdepth}{2}


%------Logo
% \setlogo[pics/logo.png] %Comment to remove the logo
\fancyfoot[LO, RE]{\includegraphics[scale=0.05]{pics/logo.png}}
%}}}1

%------Title (with default LaTeX style)
\title{Projet JEE -- Rapport}
\author{
    Équipe Tonnerre de Zeus
    \\ Laszlo \textsc{Abadie}
    \\ Adem \textsc{Benkeddad}
    \\ Samy \textsc{Chaabi}
    \\ Sami \textsc{Taki}
    \\ Louis \textsc{Thomas-Girardey}
}
\date{
    \today
    \\
    $\phantom{a}$
    \\
    \includegraphics[width=.6\textwidth]{pics/logo.png}
}

%---------------------------------Begin Document
\begin{document}
    
    % Title {{{1
    % \thetitle{Projet d'algorithmique avancée}{Rapport}
    \maketitle

    \newpage
    
    \tableofcontents
    \listoffigures
    \listoftables
    \listofalgorithms
    \newpage
    % }}}1

    \begin{indt}{\section{Structure du projet}} %{{{1
        \begin{indt}{\subsection{Arborescence}} %{{{2
            \begin{figure}[H]
                \centering
            
                \includegraphics[scale=.7]{pics/tree_0.png}
            
                \caption{Arborescence du projet (fichiers java)}
            \end{figure}

            Le projet est découpé en plusieurs packages java.
        \end{indt} %}}}2

        \begin{indt}{\subsection{API}} %{{{2
            \begin{indt}{Nous avons choisi d'implémenter une API REST suivante, avec trois points d'entrée :}
                $-$ \texttt{/api/attractions} ;

                $-$ \texttt{/api/restaurants} ;

                $-$ \texttt{/api/shows} ;
            \end{indt}

            Avec par exemple pour les attractions :

            \begin{lstlisting}[xleftmargin=60pt]
GET /api/attractions         - Retreives a list of all attractions
GET /api/attractions/{id}    - Retreives data for the attraction of id `id`
POST /api/attractions        - Creates a new attraction
PUT /api/attractions/{id}    - Updates an existing attraction
DELETE /api/attractions/{id} - Deletes an existing attraction \end{lstlisting}
        \end{indt} %}}}2
    \end{indt} %}}}1

    \begin{indt}{\section{Choix d'implémentation}} %{{{1
        Il y a trois classes principales : \texttt{Attraction}, \texttt{Restaurant} et \texttt{Show}.
        Ces classes représentent respectivement les attractions, restaurants et les spectacles.

        En plus de ces classes, on a pour chacune une classe DAOImpl, qui contient des méthodes permettant d'échanger avec la base de données.
        Il y a également pour chaque classe une classe Controller qui permet de créer les endpoints.

        Il y a une interface \texttt{GeneralDAO} qui est paramétrée par un type (qui sera une des classes), qui est implémentée par les DAOImpl.
        Elle contient aussi des méthodes statiques.

        \vspace{12pt}
        
        Nous avons choisi de d'implémenter la logique (\textit{i.e} les vérifications d'horaires, de chevauchement, etc.) dans des triggers sur la base de donnée.

        \vspace{12pt}
        
        Pour le frontend, nous avons factorisé le code javascript de génération des pages HTML, comme elles sont très similaires.
    \end{indt} %}}}1

    \begin{indt}{\section{État du projet}} %{{{1
        Nous avons un backend fonctionnel : exposition de l'API, communication entre le java et le SQL, les triggers.

        Le frontend est partiellement fonctionnel, mais ne communique pas bien avec le backend.
    \end{indt} %}}}1
    
\end{document}
%--------------------------------------------End

% vim:foldmethod=marker:foldlevel=0
\documentclass[a4paper, 12pt, twoside]{article}


%------------------------------------------------------------------------
%
% Author                :   Lasercata
% Last modification     :   2025.05.07
%
%------------------------------------------------------------------------

%---------Init {{{1
%------Lang
\usepackage[french]{babel}
%\usepackage[english]{babel}


%See https://github.com/lasercata/LaTeX_Templates for the file latex_base.sty
% \input{~/Templates/latex_base.sty}
\input{latex_base.sty}


%------Circuitikz
%\usetikzlibrary{babel}             %Uncomment this to use circuitikz
%\usetikzlibrary{shapes.geometric}  % To draw triangles in trees
%\usepackage[european]{circuitikz}            %Electrical circuits drawing

%------Sections
%---To change section numbering :
% \renewcommand\thesection{\Roman{section}}
% \renewcommand\thesubsection{\arabic{subsection})}
% \renewcommand\thesubsubsection{\textit \alph{subsubsection})}

%---To start numbering sections from 0
% \setcounter{section}{-1}

%---To hide subsubsection from the table of contents (show with max depth of 2)
% \setcounter{tocdepth}{2}


%------Logo
% \setlogo[pics/logo.png] %Comment to remove the logo
\fancyfoot[LO, RE]{\includegraphics[scale=0.05]{pics/logo.png}}
%}}}1

%------Title (with default LaTeX style)
\title{Projet JEE -- Rapport}
\author{
    Équipe Tonnerre de Zeus
    \\ Laszlo \textsc{Abadie}
    \\ Adem \textsc{Benkeddad}
    \\ Samy \textsc{Chaabi}
    \\ Sami \textsc{Taki}
    \\ Louis \textsc{Thomas-Girardey}
}
\date{
    \today
    \\
    $\phantom{a}$
    \\
    \includegraphics[width=.6\textwidth]{pics/logo.png}
}

%---------------------------------Begin Document
\begin{document}
    
    % Title {{{1
    % \thetitle{Projet d'algorithmique avancée}{Rapport}
    \maketitle

    \newpage
    
    \tableofcontents
    \listoffigures
    \listoftables
    \listofalgorithms
    \newpage
    % }}}1

    \begin{indt}{\section{Structure du projet}} %{{{1
        \begin{indt}{\subsection{Arborescence}} %{{{2
            \begin{figure}[H]
                \centering
            
                \includegraphics[scale=.7]{pics/tree_0.png}
            
                \caption{Arborescence du projet (fichiers java)}
            \end{figure}

            Le projet est découpé en plusieurs packages java.
        \end{indt} %}}}2

        \begin{indt}{\subsection{API}} %{{{2
            \begin{indt}{Nous avons choisi d'implémenter une API REST suivante, avec trois points d'entrée :}
                $-$ \texttt{/api/attractions} ;

                $-$ \texttt{/api/restaurants} ;

                $-$ \texttt{/api/shows} ;
            \end{indt}

            Avec par exemple pour les attractions :

            \begin{lstlisting}[xleftmargin=60pt]
GET /api/attractions         - Retreives a list of all attractions
GET /api/attractions/{id}    - Retreives data for the attraction of id `id`
POST /api/attractions        - Creates a new attraction
PUT /api/attractions/{id}    - Updates an existing attraction
DELETE /api/attractions/{id} - Deletes an existing attraction \end{lstlisting}
        \end{indt} %}}}2
    \end{indt} %}}}1

    \begin{indt}{\section{Choix d'implémentation}} %{{{1
        Il y a trois classes principales : \texttt{Attraction}, \texttt{Restaurant} et \texttt{Show}.
        Ces classes représentent respectivement les attractions, restaurants et les spectacles.

        En plus de ces classes, on a pour chacune une classe DAOImpl, qui contient des méthodes permettant d'échanger avec la base de données.
        Il y a également pour chaque classe une classe Controller qui permet de créer les endpoints.

        Il y a une interface \texttt{GeneralDAO} qui est paramétrée par un type (qui sera une des classes), qui est implémentée par les DAOImpl.
        Elle contient aussi des méthodes statiques.

        \vspace{12pt}
        
        Nous avons choisi de d'implémenter la logique (\textit{i.e} les vérifications d'horaires, de chevauchement, etc.) dans des triggers sur la base de donnée.

        \vspace{12pt}
        
        Pour le frontend, nous avons factorisé le code javascript de génération des pages HTML, comme elles sont très similaires.
    \end{indt} %}}}1

    \begin{indt}{\section{État du projet}} %{{{1
        Nous avons un backend fonctionnel : exposition de l'API, communication entre le java et le SQL, les triggers.

        Le frontend est partiellement fonctionnel, mais ne communique pas bien avec le backend.
    \end{indt} %}}}1
    
\end{document}
%--------------------------------------------End

% vim:foldmethod=marker:foldlevel=0


%------Circuitikz
%\usetikzlibrary{babel}             %Uncomment this to use circuitikz
%\usetikzlibrary{shapes.geometric}  % To draw triangles in trees
%\usepackage[european]{circuitikz}            %Electrical circuits drawing

%------Sections
%---To change section numbering :
% \renewcommand\thesection{\Roman{section}}
% \renewcommand\thesubsection{\arabic{subsection})}
% \renewcommand\thesubsubsection{\textit \alph{subsubsection})}

%---To start numbering sections from 0
% \setcounter{section}{-1}

%---To hide subsubsection from the table of contents (show with max depth of 2)
% \setcounter{tocdepth}{2}


%------Logo
% \setlogo[pics/logo.png] %Comment to remove the logo
\fancyfoot[LO, RE]{\includegraphics[scale=0.05]{pics/logo.png}}
%}}}1

%------Title (with default LaTeX style)
\title{Projet JEE -- Rapport}
\author{
    Équipe Tonnerre de Zeus
    \\ Laszlo \textsc{Abadie}
    \\ Adem \textsc{Benkeddad}
    \\ Samy \textsc{Chaabi}
    \\ Sami \textsc{Taki}
    \\ Louis \textsc{Thomas-Girardey}
}
\date{
    \today
    \\
    $\phantom{a}$
    \\
    \includegraphics[width=.6\textwidth]{pics/logo.png}
}

%---------------------------------Begin Document
\begin{document}
    
    % Title {{{1
    % \thetitle{Projet d'algorithmique avancée}{Rapport}
    \maketitle

    \newpage
    
    \tableofcontents
    \listoffigures
    \listoftables
    \listofalgorithms
    \newpage
    % }}}1

    \begin{indt}{\section{Structure du projet}} %{{{1
        \begin{indt}{\subsection{Arborescence}} %{{{2
            \begin{figure}[H]
                \centering
            
                \includegraphics[scale=.7]{pics/tree_0.png}
            
                \caption{Arborescence du projet (fichiers java)}
            \end{figure}

            Le projet est découpé en plusieurs packages java.
        \end{indt} %}}}2

        \begin{indt}{\subsection{API}} %{{{2
            \begin{indt}{Nous avons choisi d'implémenter une API REST suivante, avec trois points d'entrée :}
                $-$ \texttt{/api/attractions} ;

                $-$ \texttt{/api/restaurants} ;

                $-$ \texttt{/api/shows} ;
            \end{indt}

            Avec par exemple pour les attractions :

            \begin{lstlisting}[xleftmargin=60pt]
GET /api/attractions         - Retreives a list of all attractions
GET /api/attractions/{id}    - Retreives data for the attraction of id `id`
POST /api/attractions        - Creates a new attraction
PUT /api/attractions/{id}    - Updates an existing attraction
DELETE /api/attractions/{id} - Deletes an existing attraction \end{lstlisting}
        \end{indt} %}}}2
    \end{indt} %}}}1

    \begin{indt}{\section{Choix d'implémentation}} %{{{1
        Il y a trois classes principales : \texttt{Attraction}, \texttt{Restaurant} et \texttt{Show}.
        Ces classes représentent respectivement les attractions, restaurants et les spectacles.

        En plus de ces classes, on a pour chacune une classe DAOImpl, qui contient des méthodes permettant d'échanger avec la base de données.
        Il y a également pour chaque classe une classe Controller qui permet de créer les endpoints.

        Il y a une interface \texttt{GeneralDAO} qui est paramétrée par un type (qui sera une des classes), qui est implémentée par les DAOImpl.
        Elle contient aussi des méthodes statiques.

        \vspace{12pt}
        
        Nous avons choisi de d'implémenter la logique (\textit{i.e} les vérifications d'horaires, de chevauchement, etc.) dans des triggers sur la base de donnée.

        \vspace{12pt}
        
        Pour le frontend, nous avons factorisé le code javascript de génération des pages HTML, comme elles sont très similaires.
    \end{indt} %}}}1

    \begin{indt}{\section{État du projet}} %{{{1
        Nous avons un backend fonctionnel : exposition de l'API, communication entre le java et le SQL, les triggers.

        Le frontend est partiellement fonctionnel, mais ne communique pas bien avec le backend.
    \end{indt} %}}}1
    
\end{document}
%--------------------------------------------End

% vim:foldmethod=marker:foldlevel=0


%------Circuitikz
%\usetikzlibrary{babel}             %Uncomment this to use circuitikz
%\usetikzlibrary{shapes.geometric}  % To draw triangles in trees
%\usepackage[european]{circuitikz}            %Electrical circuits drawing

%------Sections
%---To change section numbering :
% \renewcommand\thesection{\Roman{section}}
% \renewcommand\thesubsection{\arabic{subsection})}
% \renewcommand\thesubsubsection{\textit \alph{subsubsection})}

%---To start numbering sections from 0
% \setcounter{section}{-1}

%---To hide subsubsection from the table of contents (show with max depth of 2)
% \setcounter{tocdepth}{2}


%------Logo
% \setlogo[pics/logo.png] %Comment to remove the logo
\fancyfoot[LO, RE]{\includegraphics[scale=0.05]{pics/logo.png}}
%}}}1

%------Title (with default LaTeX style)
\title{Projet JEE -- Rapport}
\author{
    Équipe Tonnerre de Zeus
    \\ Laszlo \textsc{Abadie}
    \\ Adem \textsc{Benkeddad}
    \\ Samy \textsc{Chaabi}
    \\ Sami \textsc{Taki}
    \\ Louis \textsc{Thomas-Girardey}
}
\date{
    \today
    \\
    $\phantom{a}$
    \\
    \includegraphics[width=.6\textwidth]{pics/logo.png}
}

%---------------------------------Begin Document
\begin{document}
    
    % Title {{{1
    % \thetitle{Projet d'algorithmique avancée}{Rapport}
    \maketitle

    \newpage
    
    \tableofcontents
    \listoffigures
    \listoftables
    \listofalgorithms
    \newpage
    % }}}1

    \begin{indt}{\section{Structure du projet}} %{{{1
        \begin{indt}{\subsection{Arborescence}} %{{{2
            \begin{figure}[H]
                \centering
            
                \includegraphics[scale=.7]{pics/tree_0.png}
            
                \caption{Arborescence du projet (fichiers java)}
            \end{figure}

            Le projet est découpé en plusieurs packages java.
        \end{indt} %}}}2

        \begin{indt}{\subsection{API}} %{{{2
            \begin{indt}{Nous avons choisi d'implémenter une API REST suivante, avec trois points d'entrée :}
                $-$ \texttt{/api/attractions} ;

                $-$ \texttt{/api/restaurants} ;

                $-$ \texttt{/api/shows} ;
            \end{indt}

            Avec par exemple pour les attractions :

            \begin{lstlisting}[xleftmargin=60pt]
GET /api/attractions         - Retreives a list of all attractions
GET /api/attractions/{id}    - Retreives data for the attraction of id `id`
POST /api/attractions        - Creates a new attraction
PUT /api/attractions/{id}    - Updates an existing attraction
DELETE /api/attractions/{id} - Deletes an existing attraction \end{lstlisting}
        \end{indt} %}}}2
    \end{indt} %}}}1

    \begin{indt}{\section{Choix d'implémentation}} %{{{1
        Il y a trois classes principales : \texttt{Attraction}, \texttt{Restaurant} et \texttt{Show}.
        Ces classes représentent respectivement les attractions, restaurants et les spectacles.

        En plus de ces classes, on a pour chacune une classe DAOImpl, qui contient des méthodes permettant d'échanger avec la base de données.
        Il y a également pour chaque classe une classe Controller qui permet de créer les endpoints.

        Il y a une interface \texttt{GeneralDAO} qui est paramétrée par un type (qui sera une des classes), qui est implémentée par les DAOImpl.
        Elle contient aussi des méthodes statiques.

        \vspace{12pt}
        
        Nous avons choisi de d'implémenter la logique (\textit{i.e} les vérifications d'horaires, de chevauchement, etc.) dans des triggers sur la base de donnée.

        \vspace{12pt}
        
        Pour le frontend, nous avons factorisé le code javascript de génération des pages HTML, comme elles sont très similaires.
    \end{indt} %}}}1

    \begin{indt}{\section{État du projet}} %{{{1
        Nous avons un backend fonctionnel : exposition de l'API, communication entre le java et le SQL, les triggers.

        Le frontend est partiellement fonctionnel, mais ne communique pas bien avec le backend.
    \end{indt} %}}}1
    
\end{document}
%--------------------------------------------End

% vim:foldmethod=marker:foldlevel=0
\documentclass[a4paper, 12pt, twoside]{article}


%------------------------------------------------------------------------
%
% Author                :   Lasercata
% Last modification     :   2025.05.07
%
%------------------------------------------------------------------------

%---------Init {{{1
%------Lang
\usepackage[french]{babel}
%\usepackage[english]{babel}


%See https://github.com/lasercata/LaTeX_Templates for the file latex_base.sty
% \documentclass[a4paper, 12pt, twoside]{article}


%------------------------------------------------------------------------
%
% Author                :   Lasercata
% Last modification     :   2025.05.07
%
%------------------------------------------------------------------------

%---------Init {{{1
%------Lang
\usepackage[french]{babel}
%\usepackage[english]{babel}


%See https://github.com/lasercata/LaTeX_Templates for the file latex_base.sty
% \documentclass[a4paper, 12pt, twoside]{article}


%------------------------------------------------------------------------
%
% Author                :   Lasercata
% Last modification     :   2025.05.07
%
%------------------------------------------------------------------------

%---------Init {{{1
%------Lang
\usepackage[french]{babel}
%\usepackage[english]{babel}


%See https://github.com/lasercata/LaTeX_Templates for the file latex_base.sty
% \input{~/Templates/latex_base.sty}
\input{latex_base.sty}


%------Circuitikz
%\usetikzlibrary{babel}             %Uncomment this to use circuitikz
%\usetikzlibrary{shapes.geometric}  % To draw triangles in trees
%\usepackage[european]{circuitikz}            %Electrical circuits drawing

%------Sections
%---To change section numbering :
% \renewcommand\thesection{\Roman{section}}
% \renewcommand\thesubsection{\arabic{subsection})}
% \renewcommand\thesubsubsection{\textit \alph{subsubsection})}

%---To start numbering sections from 0
% \setcounter{section}{-1}

%---To hide subsubsection from the table of contents (show with max depth of 2)
% \setcounter{tocdepth}{2}


%------Logo
% \setlogo[pics/logo.png] %Comment to remove the logo
\fancyfoot[LO, RE]{\includegraphics[scale=0.05]{pics/logo.png}}
%}}}1

%------Title (with default LaTeX style)
\title{Projet JEE -- Rapport}
\author{
    Équipe Tonnerre de Zeus
    \\ Laszlo \textsc{Abadie}
    \\ Adem \textsc{Benkeddad}
    \\ Samy \textsc{Chaabi}
    \\ Sami \textsc{Taki}
    \\ Louis \textsc{Thomas-Girardey}
}
\date{
    \today
    \\
    $\phantom{a}$
    \\
    \includegraphics[width=.6\textwidth]{pics/logo.png}
}

%---------------------------------Begin Document
\begin{document}
    
    % Title {{{1
    % \thetitle{Projet d'algorithmique avancée}{Rapport}
    \maketitle

    \newpage
    
    \tableofcontents
    \listoffigures
    \listoftables
    \listofalgorithms
    \newpage
    % }}}1

    \begin{indt}{\section{Structure du projet}} %{{{1
        \begin{indt}{\subsection{Arborescence}} %{{{2
            \begin{figure}[H]
                \centering
            
                \includegraphics[scale=.7]{pics/tree_0.png}
            
                \caption{Arborescence du projet (fichiers java)}
            \end{figure}

            Le projet est découpé en plusieurs packages java.
        \end{indt} %}}}2

        \begin{indt}{\subsection{API}} %{{{2
            \begin{indt}{Nous avons choisi d'implémenter une API REST suivante, avec trois points d'entrée :}
                $-$ \texttt{/api/attractions} ;

                $-$ \texttt{/api/restaurants} ;

                $-$ \texttt{/api/shows} ;
            \end{indt}

            Avec par exemple pour les attractions :

            \begin{lstlisting}[xleftmargin=60pt]
GET /api/attractions         - Retreives a list of all attractions
GET /api/attractions/{id}    - Retreives data for the attraction of id `id`
POST /api/attractions        - Creates a new attraction
PUT /api/attractions/{id}    - Updates an existing attraction
DELETE /api/attractions/{id} - Deletes an existing attraction \end{lstlisting}
        \end{indt} %}}}2
    \end{indt} %}}}1

    \begin{indt}{\section{Choix d'implémentation}} %{{{1
        Il y a trois classes principales : \texttt{Attraction}, \texttt{Restaurant} et \texttt{Show}.
        Ces classes représentent respectivement les attractions, restaurants et les spectacles.

        En plus de ces classes, on a pour chacune une classe DAOImpl, qui contient des méthodes permettant d'échanger avec la base de données.
        Il y a également pour chaque classe une classe Controller qui permet de créer les endpoints.

        Il y a une interface \texttt{GeneralDAO} qui est paramétrée par un type (qui sera une des classes), qui est implémentée par les DAOImpl.
        Elle contient aussi des méthodes statiques.

        \vspace{12pt}
        
        Nous avons choisi de d'implémenter la logique (\textit{i.e} les vérifications d'horaires, de chevauchement, etc.) dans des triggers sur la base de donnée.

        \vspace{12pt}
        
        Pour le frontend, nous avons factorisé le code javascript de génération des pages HTML, comme elles sont très similaires.
    \end{indt} %}}}1

    \begin{indt}{\section{État du projet}} %{{{1
        Nous avons un backend fonctionnel : exposition de l'API, communication entre le java et le SQL, les triggers.

        Le frontend est partiellement fonctionnel, mais ne communique pas bien avec le backend.
    \end{indt} %}}}1
    
\end{document}
%--------------------------------------------End

% vim:foldmethod=marker:foldlevel=0
\documentclass[a4paper, 12pt, twoside]{article}


%------------------------------------------------------------------------
%
% Author                :   Lasercata
% Last modification     :   2025.05.07
%
%------------------------------------------------------------------------

%---------Init {{{1
%------Lang
\usepackage[french]{babel}
%\usepackage[english]{babel}


%See https://github.com/lasercata/LaTeX_Templates for the file latex_base.sty
% \input{~/Templates/latex_base.sty}
\input{latex_base.sty}


%------Circuitikz
%\usetikzlibrary{babel}             %Uncomment this to use circuitikz
%\usetikzlibrary{shapes.geometric}  % To draw triangles in trees
%\usepackage[european]{circuitikz}            %Electrical circuits drawing

%------Sections
%---To change section numbering :
% \renewcommand\thesection{\Roman{section}}
% \renewcommand\thesubsection{\arabic{subsection})}
% \renewcommand\thesubsubsection{\textit \alph{subsubsection})}

%---To start numbering sections from 0
% \setcounter{section}{-1}

%---To hide subsubsection from the table of contents (show with max depth of 2)
% \setcounter{tocdepth}{2}


%------Logo
% \setlogo[pics/logo.png] %Comment to remove the logo
\fancyfoot[LO, RE]{\includegraphics[scale=0.05]{pics/logo.png}}
%}}}1

%------Title (with default LaTeX style)
\title{Projet JEE -- Rapport}
\author{
    Équipe Tonnerre de Zeus
    \\ Laszlo \textsc{Abadie}
    \\ Adem \textsc{Benkeddad}
    \\ Samy \textsc{Chaabi}
    \\ Sami \textsc{Taki}
    \\ Louis \textsc{Thomas-Girardey}
}
\date{
    \today
    \\
    $\phantom{a}$
    \\
    \includegraphics[width=.6\textwidth]{pics/logo.png}
}

%---------------------------------Begin Document
\begin{document}
    
    % Title {{{1
    % \thetitle{Projet d'algorithmique avancée}{Rapport}
    \maketitle

    \newpage
    
    \tableofcontents
    \listoffigures
    \listoftables
    \listofalgorithms
    \newpage
    % }}}1

    \begin{indt}{\section{Structure du projet}} %{{{1
        \begin{indt}{\subsection{Arborescence}} %{{{2
            \begin{figure}[H]
                \centering
            
                \includegraphics[scale=.7]{pics/tree_0.png}
            
                \caption{Arborescence du projet (fichiers java)}
            \end{figure}

            Le projet est découpé en plusieurs packages java.
        \end{indt} %}}}2

        \begin{indt}{\subsection{API}} %{{{2
            \begin{indt}{Nous avons choisi d'implémenter une API REST suivante, avec trois points d'entrée :}
                $-$ \texttt{/api/attractions} ;

                $-$ \texttt{/api/restaurants} ;

                $-$ \texttt{/api/shows} ;
            \end{indt}

            Avec par exemple pour les attractions :

            \begin{lstlisting}[xleftmargin=60pt]
GET /api/attractions         - Retreives a list of all attractions
GET /api/attractions/{id}    - Retreives data for the attraction of id `id`
POST /api/attractions        - Creates a new attraction
PUT /api/attractions/{id}    - Updates an existing attraction
DELETE /api/attractions/{id} - Deletes an existing attraction \end{lstlisting}
        \end{indt} %}}}2
    \end{indt} %}}}1

    \begin{indt}{\section{Choix d'implémentation}} %{{{1
        Il y a trois classes principales : \texttt{Attraction}, \texttt{Restaurant} et \texttt{Show}.
        Ces classes représentent respectivement les attractions, restaurants et les spectacles.

        En plus de ces classes, on a pour chacune une classe DAOImpl, qui contient des méthodes permettant d'échanger avec la base de données.
        Il y a également pour chaque classe une classe Controller qui permet de créer les endpoints.

        Il y a une interface \texttt{GeneralDAO} qui est paramétrée par un type (qui sera une des classes), qui est implémentée par les DAOImpl.
        Elle contient aussi des méthodes statiques.

        \vspace{12pt}
        
        Nous avons choisi de d'implémenter la logique (\textit{i.e} les vérifications d'horaires, de chevauchement, etc.) dans des triggers sur la base de donnée.

        \vspace{12pt}
        
        Pour le frontend, nous avons factorisé le code javascript de génération des pages HTML, comme elles sont très similaires.
    \end{indt} %}}}1

    \begin{indt}{\section{État du projet}} %{{{1
        Nous avons un backend fonctionnel : exposition de l'API, communication entre le java et le SQL, les triggers.

        Le frontend est partiellement fonctionnel, mais ne communique pas bien avec le backend.
    \end{indt} %}}}1
    
\end{document}
%--------------------------------------------End

% vim:foldmethod=marker:foldlevel=0


%------Circuitikz
%\usetikzlibrary{babel}             %Uncomment this to use circuitikz
%\usetikzlibrary{shapes.geometric}  % To draw triangles in trees
%\usepackage[european]{circuitikz}            %Electrical circuits drawing

%------Sections
%---To change section numbering :
% \renewcommand\thesection{\Roman{section}}
% \renewcommand\thesubsection{\arabic{subsection})}
% \renewcommand\thesubsubsection{\textit \alph{subsubsection})}

%---To start numbering sections from 0
% \setcounter{section}{-1}

%---To hide subsubsection from the table of contents (show with max depth of 2)
% \setcounter{tocdepth}{2}


%------Logo
% \setlogo[pics/logo.png] %Comment to remove the logo
\fancyfoot[LO, RE]{\includegraphics[scale=0.05]{pics/logo.png}}
%}}}1

%------Title (with default LaTeX style)
\title{Projet JEE -- Rapport}
\author{
    Équipe Tonnerre de Zeus
    \\ Laszlo \textsc{Abadie}
    \\ Adem \textsc{Benkeddad}
    \\ Samy \textsc{Chaabi}
    \\ Sami \textsc{Taki}
    \\ Louis \textsc{Thomas-Girardey}
}
\date{
    \today
    \\
    $\phantom{a}$
    \\
    \includegraphics[width=.6\textwidth]{pics/logo.png}
}

%---------------------------------Begin Document
\begin{document}
    
    % Title {{{1
    % \thetitle{Projet d'algorithmique avancée}{Rapport}
    \maketitle

    \newpage
    
    \tableofcontents
    \listoffigures
    \listoftables
    \listofalgorithms
    \newpage
    % }}}1

    \begin{indt}{\section{Structure du projet}} %{{{1
        \begin{indt}{\subsection{Arborescence}} %{{{2
            \begin{figure}[H]
                \centering
            
                \includegraphics[scale=.7]{pics/tree_0.png}
            
                \caption{Arborescence du projet (fichiers java)}
            \end{figure}

            Le projet est découpé en plusieurs packages java.
        \end{indt} %}}}2

        \begin{indt}{\subsection{API}} %{{{2
            \begin{indt}{Nous avons choisi d'implémenter une API REST suivante, avec trois points d'entrée :}
                $-$ \texttt{/api/attractions} ;

                $-$ \texttt{/api/restaurants} ;

                $-$ \texttt{/api/shows} ;
            \end{indt}

            Avec par exemple pour les attractions :

            \begin{lstlisting}[xleftmargin=60pt]
GET /api/attractions         - Retreives a list of all attractions
GET /api/attractions/{id}    - Retreives data for the attraction of id `id`
POST /api/attractions        - Creates a new attraction
PUT /api/attractions/{id}    - Updates an existing attraction
DELETE /api/attractions/{id} - Deletes an existing attraction \end{lstlisting}
        \end{indt} %}}}2
    \end{indt} %}}}1

    \begin{indt}{\section{Choix d'implémentation}} %{{{1
        Il y a trois classes principales : \texttt{Attraction}, \texttt{Restaurant} et \texttt{Show}.
        Ces classes représentent respectivement les attractions, restaurants et les spectacles.

        En plus de ces classes, on a pour chacune une classe DAOImpl, qui contient des méthodes permettant d'échanger avec la base de données.
        Il y a également pour chaque classe une classe Controller qui permet de créer les endpoints.

        Il y a une interface \texttt{GeneralDAO} qui est paramétrée par un type (qui sera une des classes), qui est implémentée par les DAOImpl.
        Elle contient aussi des méthodes statiques.

        \vspace{12pt}
        
        Nous avons choisi de d'implémenter la logique (\textit{i.e} les vérifications d'horaires, de chevauchement, etc.) dans des triggers sur la base de donnée.

        \vspace{12pt}
        
        Pour le frontend, nous avons factorisé le code javascript de génération des pages HTML, comme elles sont très similaires.
    \end{indt} %}}}1

    \begin{indt}{\section{État du projet}} %{{{1
        Nous avons un backend fonctionnel : exposition de l'API, communication entre le java et le SQL, les triggers.

        Le frontend est partiellement fonctionnel, mais ne communique pas bien avec le backend.
    \end{indt} %}}}1
    
\end{document}
%--------------------------------------------End

% vim:foldmethod=marker:foldlevel=0
\documentclass[a4paper, 12pt, twoside]{article}


%------------------------------------------------------------------------
%
% Author                :   Lasercata
% Last modification     :   2025.05.07
%
%------------------------------------------------------------------------

%---------Init {{{1
%------Lang
\usepackage[french]{babel}
%\usepackage[english]{babel}


%See https://github.com/lasercata/LaTeX_Templates for the file latex_base.sty
% \documentclass[a4paper, 12pt, twoside]{article}


%------------------------------------------------------------------------
%
% Author                :   Lasercata
% Last modification     :   2025.05.07
%
%------------------------------------------------------------------------

%---------Init {{{1
%------Lang
\usepackage[french]{babel}
%\usepackage[english]{babel}


%See https://github.com/lasercata/LaTeX_Templates for the file latex_base.sty
% \input{~/Templates/latex_base.sty}
\input{latex_base.sty}


%------Circuitikz
%\usetikzlibrary{babel}             %Uncomment this to use circuitikz
%\usetikzlibrary{shapes.geometric}  % To draw triangles in trees
%\usepackage[european]{circuitikz}            %Electrical circuits drawing

%------Sections
%---To change section numbering :
% \renewcommand\thesection{\Roman{section}}
% \renewcommand\thesubsection{\arabic{subsection})}
% \renewcommand\thesubsubsection{\textit \alph{subsubsection})}

%---To start numbering sections from 0
% \setcounter{section}{-1}

%---To hide subsubsection from the table of contents (show with max depth of 2)
% \setcounter{tocdepth}{2}


%------Logo
% \setlogo[pics/logo.png] %Comment to remove the logo
\fancyfoot[LO, RE]{\includegraphics[scale=0.05]{pics/logo.png}}
%}}}1

%------Title (with default LaTeX style)
\title{Projet JEE -- Rapport}
\author{
    Équipe Tonnerre de Zeus
    \\ Laszlo \textsc{Abadie}
    \\ Adem \textsc{Benkeddad}
    \\ Samy \textsc{Chaabi}
    \\ Sami \textsc{Taki}
    \\ Louis \textsc{Thomas-Girardey}
}
\date{
    \today
    \\
    $\phantom{a}$
    \\
    \includegraphics[width=.6\textwidth]{pics/logo.png}
}

%---------------------------------Begin Document
\begin{document}
    
    % Title {{{1
    % \thetitle{Projet d'algorithmique avancée}{Rapport}
    \maketitle

    \newpage
    
    \tableofcontents
    \listoffigures
    \listoftables
    \listofalgorithms
    \newpage
    % }}}1

    \begin{indt}{\section{Structure du projet}} %{{{1
        \begin{indt}{\subsection{Arborescence}} %{{{2
            \begin{figure}[H]
                \centering
            
                \includegraphics[scale=.7]{pics/tree_0.png}
            
                \caption{Arborescence du projet (fichiers java)}
            \end{figure}

            Le projet est découpé en plusieurs packages java.
        \end{indt} %}}}2

        \begin{indt}{\subsection{API}} %{{{2
            \begin{indt}{Nous avons choisi d'implémenter une API REST suivante, avec trois points d'entrée :}
                $-$ \texttt{/api/attractions} ;

                $-$ \texttt{/api/restaurants} ;

                $-$ \texttt{/api/shows} ;
            \end{indt}

            Avec par exemple pour les attractions :

            \begin{lstlisting}[xleftmargin=60pt]
GET /api/attractions         - Retreives a list of all attractions
GET /api/attractions/{id}    - Retreives data for the attraction of id `id`
POST /api/attractions        - Creates a new attraction
PUT /api/attractions/{id}    - Updates an existing attraction
DELETE /api/attractions/{id} - Deletes an existing attraction \end{lstlisting}
        \end{indt} %}}}2
    \end{indt} %}}}1

    \begin{indt}{\section{Choix d'implémentation}} %{{{1
        Il y a trois classes principales : \texttt{Attraction}, \texttt{Restaurant} et \texttt{Show}.
        Ces classes représentent respectivement les attractions, restaurants et les spectacles.

        En plus de ces classes, on a pour chacune une classe DAOImpl, qui contient des méthodes permettant d'échanger avec la base de données.
        Il y a également pour chaque classe une classe Controller qui permet de créer les endpoints.

        Il y a une interface \texttt{GeneralDAO} qui est paramétrée par un type (qui sera une des classes), qui est implémentée par les DAOImpl.
        Elle contient aussi des méthodes statiques.

        \vspace{12pt}
        
        Nous avons choisi de d'implémenter la logique (\textit{i.e} les vérifications d'horaires, de chevauchement, etc.) dans des triggers sur la base de donnée.

        \vspace{12pt}
        
        Pour le frontend, nous avons factorisé le code javascript de génération des pages HTML, comme elles sont très similaires.
    \end{indt} %}}}1

    \begin{indt}{\section{État du projet}} %{{{1
        Nous avons un backend fonctionnel : exposition de l'API, communication entre le java et le SQL, les triggers.

        Le frontend est partiellement fonctionnel, mais ne communique pas bien avec le backend.
    \end{indt} %}}}1
    
\end{document}
%--------------------------------------------End

% vim:foldmethod=marker:foldlevel=0
\documentclass[a4paper, 12pt, twoside]{article}


%------------------------------------------------------------------------
%
% Author                :   Lasercata
% Last modification     :   2025.05.07
%
%------------------------------------------------------------------------

%---------Init {{{1
%------Lang
\usepackage[french]{babel}
%\usepackage[english]{babel}


%See https://github.com/lasercata/LaTeX_Templates for the file latex_base.sty
% \input{~/Templates/latex_base.sty}
\input{latex_base.sty}


%------Circuitikz
%\usetikzlibrary{babel}             %Uncomment this to use circuitikz
%\usetikzlibrary{shapes.geometric}  % To draw triangles in trees
%\usepackage[european]{circuitikz}            %Electrical circuits drawing

%------Sections
%---To change section numbering :
% \renewcommand\thesection{\Roman{section}}
% \renewcommand\thesubsection{\arabic{subsection})}
% \renewcommand\thesubsubsection{\textit \alph{subsubsection})}

%---To start numbering sections from 0
% \setcounter{section}{-1}

%---To hide subsubsection from the table of contents (show with max depth of 2)
% \setcounter{tocdepth}{2}


%------Logo
% \setlogo[pics/logo.png] %Comment to remove the logo
\fancyfoot[LO, RE]{\includegraphics[scale=0.05]{pics/logo.png}}
%}}}1

%------Title (with default LaTeX style)
\title{Projet JEE -- Rapport}
\author{
    Équipe Tonnerre de Zeus
    \\ Laszlo \textsc{Abadie}
    \\ Adem \textsc{Benkeddad}
    \\ Samy \textsc{Chaabi}
    \\ Sami \textsc{Taki}
    \\ Louis \textsc{Thomas-Girardey}
}
\date{
    \today
    \\
    $\phantom{a}$
    \\
    \includegraphics[width=.6\textwidth]{pics/logo.png}
}

%---------------------------------Begin Document
\begin{document}
    
    % Title {{{1
    % \thetitle{Projet d'algorithmique avancée}{Rapport}
    \maketitle

    \newpage
    
    \tableofcontents
    \listoffigures
    \listoftables
    \listofalgorithms
    \newpage
    % }}}1

    \begin{indt}{\section{Structure du projet}} %{{{1
        \begin{indt}{\subsection{Arborescence}} %{{{2
            \begin{figure}[H]
                \centering
            
                \includegraphics[scale=.7]{pics/tree_0.png}
            
                \caption{Arborescence du projet (fichiers java)}
            \end{figure}

            Le projet est découpé en plusieurs packages java.
        \end{indt} %}}}2

        \begin{indt}{\subsection{API}} %{{{2
            \begin{indt}{Nous avons choisi d'implémenter une API REST suivante, avec trois points d'entrée :}
                $-$ \texttt{/api/attractions} ;

                $-$ \texttt{/api/restaurants} ;

                $-$ \texttt{/api/shows} ;
            \end{indt}

            Avec par exemple pour les attractions :

            \begin{lstlisting}[xleftmargin=60pt]
GET /api/attractions         - Retreives a list of all attractions
GET /api/attractions/{id}    - Retreives data for the attraction of id `id`
POST /api/attractions        - Creates a new attraction
PUT /api/attractions/{id}    - Updates an existing attraction
DELETE /api/attractions/{id} - Deletes an existing attraction \end{lstlisting}
        \end{indt} %}}}2
    \end{indt} %}}}1

    \begin{indt}{\section{Choix d'implémentation}} %{{{1
        Il y a trois classes principales : \texttt{Attraction}, \texttt{Restaurant} et \texttt{Show}.
        Ces classes représentent respectivement les attractions, restaurants et les spectacles.

        En plus de ces classes, on a pour chacune une classe DAOImpl, qui contient des méthodes permettant d'échanger avec la base de données.
        Il y a également pour chaque classe une classe Controller qui permet de créer les endpoints.

        Il y a une interface \texttt{GeneralDAO} qui est paramétrée par un type (qui sera une des classes), qui est implémentée par les DAOImpl.
        Elle contient aussi des méthodes statiques.

        \vspace{12pt}
        
        Nous avons choisi de d'implémenter la logique (\textit{i.e} les vérifications d'horaires, de chevauchement, etc.) dans des triggers sur la base de donnée.

        \vspace{12pt}
        
        Pour le frontend, nous avons factorisé le code javascript de génération des pages HTML, comme elles sont très similaires.
    \end{indt} %}}}1

    \begin{indt}{\section{État du projet}} %{{{1
        Nous avons un backend fonctionnel : exposition de l'API, communication entre le java et le SQL, les triggers.

        Le frontend est partiellement fonctionnel, mais ne communique pas bien avec le backend.
    \end{indt} %}}}1
    
\end{document}
%--------------------------------------------End

% vim:foldmethod=marker:foldlevel=0


%------Circuitikz
%\usetikzlibrary{babel}             %Uncomment this to use circuitikz
%\usetikzlibrary{shapes.geometric}  % To draw triangles in trees
%\usepackage[european]{circuitikz}            %Electrical circuits drawing

%------Sections
%---To change section numbering :
% \renewcommand\thesection{\Roman{section}}
% \renewcommand\thesubsection{\arabic{subsection})}
% \renewcommand\thesubsubsection{\textit \alph{subsubsection})}

%---To start numbering sections from 0
% \setcounter{section}{-1}

%---To hide subsubsection from the table of contents (show with max depth of 2)
% \setcounter{tocdepth}{2}


%------Logo
% \setlogo[pics/logo.png] %Comment to remove the logo
\fancyfoot[LO, RE]{\includegraphics[scale=0.05]{pics/logo.png}}
%}}}1

%------Title (with default LaTeX style)
\title{Projet JEE -- Rapport}
\author{
    Équipe Tonnerre de Zeus
    \\ Laszlo \textsc{Abadie}
    \\ Adem \textsc{Benkeddad}
    \\ Samy \textsc{Chaabi}
    \\ Sami \textsc{Taki}
    \\ Louis \textsc{Thomas-Girardey}
}
\date{
    \today
    \\
    $\phantom{a}$
    \\
    \includegraphics[width=.6\textwidth]{pics/logo.png}
}

%---------------------------------Begin Document
\begin{document}
    
    % Title {{{1
    % \thetitle{Projet d'algorithmique avancée}{Rapport}
    \maketitle

    \newpage
    
    \tableofcontents
    \listoffigures
    \listoftables
    \listofalgorithms
    \newpage
    % }}}1

    \begin{indt}{\section{Structure du projet}} %{{{1
        \begin{indt}{\subsection{Arborescence}} %{{{2
            \begin{figure}[H]
                \centering
            
                \includegraphics[scale=.7]{pics/tree_0.png}
            
                \caption{Arborescence du projet (fichiers java)}
            \end{figure}

            Le projet est découpé en plusieurs packages java.
        \end{indt} %}}}2

        \begin{indt}{\subsection{API}} %{{{2
            \begin{indt}{Nous avons choisi d'implémenter une API REST suivante, avec trois points d'entrée :}
                $-$ \texttt{/api/attractions} ;

                $-$ \texttt{/api/restaurants} ;

                $-$ \texttt{/api/shows} ;
            \end{indt}

            Avec par exemple pour les attractions :

            \begin{lstlisting}[xleftmargin=60pt]
GET /api/attractions         - Retreives a list of all attractions
GET /api/attractions/{id}    - Retreives data for the attraction of id `id`
POST /api/attractions        - Creates a new attraction
PUT /api/attractions/{id}    - Updates an existing attraction
DELETE /api/attractions/{id} - Deletes an existing attraction \end{lstlisting}
        \end{indt} %}}}2
    \end{indt} %}}}1

    \begin{indt}{\section{Choix d'implémentation}} %{{{1
        Il y a trois classes principales : \texttt{Attraction}, \texttt{Restaurant} et \texttt{Show}.
        Ces classes représentent respectivement les attractions, restaurants et les spectacles.

        En plus de ces classes, on a pour chacune une classe DAOImpl, qui contient des méthodes permettant d'échanger avec la base de données.
        Il y a également pour chaque classe une classe Controller qui permet de créer les endpoints.

        Il y a une interface \texttt{GeneralDAO} qui est paramétrée par un type (qui sera une des classes), qui est implémentée par les DAOImpl.
        Elle contient aussi des méthodes statiques.

        \vspace{12pt}
        
        Nous avons choisi de d'implémenter la logique (\textit{i.e} les vérifications d'horaires, de chevauchement, etc.) dans des triggers sur la base de donnée.

        \vspace{12pt}
        
        Pour le frontend, nous avons factorisé le code javascript de génération des pages HTML, comme elles sont très similaires.
    \end{indt} %}}}1

    \begin{indt}{\section{État du projet}} %{{{1
        Nous avons un backend fonctionnel : exposition de l'API, communication entre le java et le SQL, les triggers.

        Le frontend est partiellement fonctionnel, mais ne communique pas bien avec le backend.
    \end{indt} %}}}1
    
\end{document}
%--------------------------------------------End

% vim:foldmethod=marker:foldlevel=0


%------Circuitikz
%\usetikzlibrary{babel}             %Uncomment this to use circuitikz
%\usetikzlibrary{shapes.geometric}  % To draw triangles in trees
%\usepackage[european]{circuitikz}            %Electrical circuits drawing

%------Sections
%---To change section numbering :
% \renewcommand\thesection{\Roman{section}}
% \renewcommand\thesubsection{\arabic{subsection})}
% \renewcommand\thesubsubsection{\textit \alph{subsubsection})}

%---To start numbering sections from 0
% \setcounter{section}{-1}

%---To hide subsubsection from the table of contents (show with max depth of 2)
% \setcounter{tocdepth}{2}


%------Logo
% \setlogo[pics/logo.png] %Comment to remove the logo
\fancyfoot[LO, RE]{\includegraphics[scale=0.05]{pics/logo.png}}
%}}}1

%------Title (with default LaTeX style)
\title{Projet JEE -- Rapport}
\author{
    Équipe Tonnerre de Zeus
    \\ Laszlo \textsc{Abadie}
    \\ Adem \textsc{Benkeddad}
    \\ Samy \textsc{Chaabi}
    \\ Sami \textsc{Taki}
    \\ Louis \textsc{Thomas-Girardey}
}
\date{
    \today
    \\
    $\phantom{a}$
    \\
    \includegraphics[width=.6\textwidth]{pics/logo.png}
}

%---------------------------------Begin Document
\begin{document}
    
    % Title {{{1
    % \thetitle{Projet d'algorithmique avancée}{Rapport}
    \maketitle

    \newpage
    
    \tableofcontents
    \listoffigures
    \listoftables
    \listofalgorithms
    \newpage
    % }}}1

    \begin{indt}{\section{Structure du projet}} %{{{1
        \begin{indt}{\subsection{Arborescence}} %{{{2
            \begin{figure}[H]
                \centering
            
                \includegraphics[scale=.7]{pics/tree_0.png}
            
                \caption{Arborescence du projet (fichiers java)}
            \end{figure}

            Le projet est découpé en plusieurs packages java.
        \end{indt} %}}}2

        \begin{indt}{\subsection{API}} %{{{2
            \begin{indt}{Nous avons choisi d'implémenter une API REST suivante, avec trois points d'entrée :}
                $-$ \texttt{/api/attractions} ;

                $-$ \texttt{/api/restaurants} ;

                $-$ \texttt{/api/shows} ;
            \end{indt}

            Avec par exemple pour les attractions :

            \begin{lstlisting}[xleftmargin=60pt]
GET /api/attractions         - Retreives a list of all attractions
GET /api/attractions/{id}    - Retreives data for the attraction of id `id`
POST /api/attractions        - Creates a new attraction
PUT /api/attractions/{id}    - Updates an existing attraction
DELETE /api/attractions/{id} - Deletes an existing attraction \end{lstlisting}
        \end{indt} %}}}2
    \end{indt} %}}}1

    \begin{indt}{\section{Choix d'implémentation}} %{{{1
        Il y a trois classes principales : \texttt{Attraction}, \texttt{Restaurant} et \texttt{Show}.
        Ces classes représentent respectivement les attractions, restaurants et les spectacles.

        En plus de ces classes, on a pour chacune une classe DAOImpl, qui contient des méthodes permettant d'échanger avec la base de données.
        Il y a également pour chaque classe une classe Controller qui permet de créer les endpoints.

        Il y a une interface \texttt{GeneralDAO} qui est paramétrée par un type (qui sera une des classes), qui est implémentée par les DAOImpl.
        Elle contient aussi des méthodes statiques.

        \vspace{12pt}
        
        Nous avons choisi de d'implémenter la logique (\textit{i.e} les vérifications d'horaires, de chevauchement, etc.) dans des triggers sur la base de donnée.

        \vspace{12pt}
        
        Pour le frontend, nous avons factorisé le code javascript de génération des pages HTML, comme elles sont très similaires.
    \end{indt} %}}}1

    \begin{indt}{\section{État du projet}} %{{{1
        Nous avons un backend fonctionnel : exposition de l'API, communication entre le java et le SQL, les triggers.

        Le frontend est partiellement fonctionnel, mais ne communique pas bien avec le backend.
    \end{indt} %}}}1
    
\end{document}
%--------------------------------------------End

% vim:foldmethod=marker:foldlevel=0


%------Circuitikz
%\usetikzlibrary{babel}             %Uncomment this to use circuitikz
%\usetikzlibrary{shapes.geometric}  % To draw triangles in trees
%\usepackage[european]{circuitikz}            %Electrical circuits drawing

%------Sections
%---To change section numbering :
% \renewcommand\thesection{\Roman{section}}
% \renewcommand\thesubsection{\arabic{subsection})}
% \renewcommand\thesubsubsection{\textit \alph{subsubsection})}

%---To start numbering sections from 0
% \setcounter{section}{-1}

%---To hide subsubsection from the table of contents (show with max depth of 2)
% \setcounter{tocdepth}{2}


%------Logo
% \setlogo[pics/logo.png] %Comment to remove the logo
% \fancyfoot[LO, RE]{\includegraphics[scale=0.05]{pics/logo.png}}

\usepackage{forest}
%}}}1

%------Title (with default LaTeX style)
\title{Wireless Project -- Report}
\author{
    \\ \NameA
    \\ \NameB
}
\date{
    \today
}

%---------------------------------Begin Document
\begin{document}
    
    % Title {{{1
    % \thetitle{}{}
    \maketitle

    \tableofcontents
    % \listoffigures
    % \listoftables
    % \listofalgorithms
    \newpage
    % }}}1

    %TODO : develop the bullet points

    \begin{indt}{\section{Project Context}} %{{{1
        This project aims to implement a receiver / decoder of a simplified 5G NR signal.
    \end{indt} %}}}1

    \begin{indt}{\section{Project architecture}}% {{{1
        \begin{indt}{\subsection{File tree}} %{{{2
            \begin{figure}[H]% {{{3
                \centering
        
                \begin{forest}
                    for tree={% {{{4
                        font=\sffamily,
                        text=black,
                        % text width=2cm,
                        % minimum height=0.75cm,
                        % if level=0
                        % {fill=ff4500}
                        % {draw=black},
                        rounded corners=4pt,
                        grow'=0,
                        child anchor=west,
                        parent anchor=south,
                        anchor=west,
                        calign=first,
                        edge={ff4500, rounded corners=1pt, line width=1pt},
                        edge path={
                            \noexpand\path [draw, \forestoption{edge}]
                            (!u.south west) +(0pt,0) |- (.child anchor)\forestoption{edge label};
                        },
                        before typesetting nodes={
                            if n=1
                            {insert before={[,phantom]}}
                            {}
                        },
                        fit=band,
                        s sep=0pt,
                        before computing xy={l=25pt},
                    }% }}}4
                    [
                        [\textcolor{00f}{report/}]
                        [\textcolor{00f}{topic/}]
                        [\textcolor{00f}{data/}
                            [tfMatrix.csv]
                            [tfMatrix\_2.csv]
                            [tfMatrix\_3.csv]
                        ]
                        [\textcolor{00f}{code/}
                            [main.py]
                            [\textcolor{00f}{src/}
                                [binary\_transformation.py]
                                [crc.py]
                                [decode.py]
                                [demod.py]
                                [hamming748.py]
                                [utils.py]
                            ]
                            [\textcolor{00f}{tests/}
                                [tests\_hamming.py]
                                [tests\_modulation.py]
                                [utils.py]
                            ]
                        ]
                        [README.sh]
                    ]
                \end{forest}
        
                \caption{Project file structure}
                \label{fig:forest}
            \end{figure}% }}}3
        \end{indt} %}}}2

        \begin{indt}{\subsection{Files description}} %{{{2
            The folder \texttt{\textcolor{00f}{data/}} contains the matrix. The matrix 1 and 3 are correct and should be correctly decoded, but the matrix 2 had a very bad transmission and cannot be recovered. 
            
            The folder \texttt{\textcolor{00f}{code/}} contains the implementation.
            
            The main file is \texttt{\textcolor{00f}{code/}main.py}. It contains a very simple command line parser to run the code with the wanted matrix.
            The usage is detailed in the readme file (and with the \texttt{-h} flag).
            
            The file \texttt{\textcolor{00f}{code/src/}utils.py} contains general functions to manage matrix (getting from file, display, flatten index, ...)
            
            The file \texttt{\textcolor{00f}{code/src/}demod.py} implements multiple demodulation functions (\texttt{bpsk}, \texttt{qpsk}, \texttt{16qam}).
            
            The file \texttt{\textcolor{00f}{code/src/}decode.py} implements the decoding of a matrix (for a given user identifier).
            It uses the other files to do so.
            It is implemented using the object oriented paradigm.
            
            The folder \texttt{\textcolor{00f}{code/tests/}} defines unit tests on hamming and modulation.
            
            The file \texttt{\textcolor{00f}{code/tests/}utils.py} useful functions for the unit tests.
        \end{indt} %}}}2
    \end{indt}% }}}1

    \begin{indt}{\section{How does it work ?}}% {{{1
        %TODO : add subsection for each big part like PBCH, PDSCH, ...
        - We retreive pbch from matrix

        - We used the bpsk to modulate the pbch because the information it contains is crucial to retreive the user data from the matrix. Indeed with only two possible values, we decrease the amount of information we can stock but this latter is way more protected and more reliable because much less noise sensitive. (insert an image of the different modulation that exist to illustrate statement)

        - We read the user \& cell ident

        - We want to retreive a particular user info : we read each pbch block until we find the one that we want, we take its starting block and then look up for it in the pdsch.

        - We do the same thing for the pdsch that gives us the payload staring block that contains the user data
    \end{indt}% }}}1

    \begin{indt}{\section{Conclusion}}% {{{1
        - Here make a brief conclusion about the project (what worked, what we learned, ...)
    \end{indt}% }}}1
    
\end{document}
%--------------------------------------------End

% vim:foldmethod=marker:foldlevel=0
